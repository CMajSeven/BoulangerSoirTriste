\documentclass[twoside]{article}
\usepackage[a3paper, margin=16mm, includehead]{geometry}
\usepackage{changepage}
\usepackage{enumitem}
\usepackage{fancyhdr}
\usepackage{fontspec}
\usepackage{graphicx}
\usepackage{hanging}
\usepackage{multicol}
\usepackage{musixtex}
\usepackage{wasysym}
\setmainfont[
 BoldFont={[ACADEMICO-BOLD.otf]}, 
 ItalicFont={[ACADEMICO-ITALIC.otf]},
 BoldItalicFont={[ACADEMICO-BOLDITALIC.otf]}
 ]{[ACADEMICO-REGULAR.otf]}
\setlength{\columnsep}{1cm}

\pagestyle{fancy}
\renewcommand{\headrulewidth}{0pt}
\fancyhf{}%clear all headers and footers
\fancyhead[LE]{\fontsize{8pt}{10pt}\selectfont \slshape\rightmark}
\fancyhead[RO]{\fontsize{8pt}{10pt}\selectfont \slshape\leftmark}
\fancyhead[LE,RO]{\thepage}
\setcounter{page}{23}

\newcommand\dynmark[1]{\scalebox{0.9}{#1}{\kern1pt}}

\begin{document}
\begin{center}
\underline{\huge{Editorial Commentary}}
\end{center}

This edition is based on manuscripts of the score and parts by a copyist from Lili Boulanger's publisher Durand. There are numerous errors and inconsistencies, and edits by a second hand. Those of greater significance are listed below. A more extensive list is available in the source repository.

\begin{hangparas}{15pt}{1}
\bigbreak
FS = Full Score
\begin{multicols}{2}

\underline{General}

1: Tempo \quarternote\ = 58 written in FS in second hand.

78: Fermata placed either on rest after best or on rest on beat depending on player. Replacing with caesura after first 8\textsuperscript{th} note

78: Based on the phrasing, recasting this as a 1+3 measure

83: ``librement'' is written like a tempo marking in FS, but is only found in the English Horn part

85: ``en récit'' placed in the middle of this measure. Moving to align with Cello Soli at 86

106--111: ``un peu serré'' and ``rit.'' are seemingly specifically placed but inconsistently across FS and parts

135--143: The intention of the crescendo hairpin to \mbox{``\dynmark{\f} \textit{mais contenu}''} above the strings in FS is not entirely clear. No indication of this or similar is found in any of the parts. ``\textit{[cresc. poco a poco]}\hspace{1pt}'' has been added to the relevant parts.

\underline{Flutes}

8--19.2: Section is crossed out in FS, but tacet is also crossed out with ``une flûte'' written over it.

19: \dynmark{\p} only in FS

37: Hairpins and tenuto marks only in part

47: The run begins a 16\textsuperscript{th} note earlier in part at beat 2¼. This allows a C♯ on beat 3 to align with the other parts.

50: Part has F\textsubscript{6}+A\textsubscript{6}. FS has D\textsubscript{6}+F\textsubscript{6}. \dynmark{\ff} in part. \dynmark{\f} in FS.

61: The run starts on beat 3 in part.

65--73: Flute 2 part is crossed out in part and FS.

115: \dynmark{\p} overwritten with \dynmark{\mf} in part

115--123: Flute 2 part only in part (rests crossed out).

171: \dynmark{\mf} only in part

177: 8\textsuperscript{th} notes crossed out in part

\underline{Oboes}

3--7: This section is crossed out in FS and part.

16--21.1: Oboe 1 part seems to be written in by second hand in FS and part

46--48: This section is written in second hand in FS and part, completely covering the original.

49: Oboe 1 C♯\textsubscript{6} originally E\textsubscript{5} in FS. It is actually the first 16\textsuperscript{th} B\textsubscript{4} of Oboe 2 that is crossed out but this makes little sense. The previous measures that were overwritten likely lead better into this E\textsubscript{5}.

50: Oboe 2 part missing in FS

66--73: Oboe 2 part crossed out in FS and part.

153--156: Tenuto marks only in part (added by second hand)

169--171: In FS, Oboe 1 plays this phrase continuously, all slurred together. Part follows orchestration of other parts more closely and is the version used in this edition.

177: 8\textsuperscript{th} note crossed out in part and FS

\underline{English Horn}

39: ♯ overwritten with a ♮ (next measure has ♯) in FS. ♮ matches Violins II and Horn 2.

47--49: This section written in second hand, covering largely illegible original. 49 is potentially unison with Horn 1.

65--67: This section is crossed out in FS.

96: ♮ on F presumably missing in FS, as the Viola Solo has B♭. However, the Viola Solo is written in a second hand in FS and part. In the part, it is covering the English Horn written as a cue and it is unclear whether the cue originally had a B(♮) or B♭.

149: Last 2½ beats crossed out by second hand. The next measures contain only whole rests with no signs of erasure, so the continuation is unclear.

153: This measure seems to have been added by a second hand in FS. Whole rest still present

\underline{Clarinets}

3: \dynmark{\mf} added (\dynmark{\p} not crossed out) in part. See also note for Bass Clarinet.

3--14: ``à 2'' with ``col 1'' added in part (``Solo'' not crossed out)

20--25.1: This section for Clarinet 1 crossed out in part and FS

23--24: This section for Clarinet 2 crossed out in FS

27.1: Quarter note F crossed out in FS (slur extends over this). Not present in part

30--33: \dynmark{\mf} in part. \dynmark{\p} in FS. Original slur (matching FS) in part broken at 31.1 and 31.3. Tenuto marks added on first 3 notes of 30 and accent added on first note of 31.

30--38: Clarinet 2 is unison with 1 in part, but ``solo'' is underlined and circled.

34: \dynmark{\f} only in part

71: Unclear in FS whether this \dynmark{\sF} should apply to Bass Clarinet or Clarinet (2). It seems more sensible to apply to Bass Clarinet but only the Clarinets part has \dynmark{\sF} (on both Clarinets), but this is perhaps a misreading.

92: \dynmark{\mf} in part. \dynmark{\p} in FS

114: ``\dynmark{\mf} \textit{en dehors}'' in part (``\textit{lointain, doux}'' not crossed out). \dynmark{\p} in FS

127: ♮ on trill only in part, added by second hand

127--128: Clarinet 2 unison with Clarinet 1 in part (8\textsuperscript{th} note written over), with slur over 126--128 and ``\textit{soutenu}'' at 127 that Clarinet 1 does not have.

153--156: \dynmark{\p} and slurred in FS. Slur erased, ``\dynmark{\mf} \textit{bien accentué}'' with accent on every note in part.

170: \dynmark{\p} with accent in part. \dynmark{\sF} in FS

\columnbreak

\underline{Bass Clarinet}

Originally written for Bass Clarinet in A, sounding a minor third below written

3: \dynmark{\mf} and \dynmark{\mezzopiano} in part. \dynmark{\p} in FS. See also note for Clarinets.

15--18.1: Section written in by second hand in FS and part

26--28: Section written in by second hand in FS and part

71: See note for Clarinets.

127: ♮ on trill only in part, added by second hand

153--156: ``\dynmark{\mf} \textit{accentué}'' with tenuto marks on every note in part. \dynmark{\p} in FS

\underline{Bassoons}

18: Bassoon 2 originally has just an 8\textsuperscript{th} note here (tied to previous measures). In FS, this is overwritten with dotted half note tied with surrounding measures.

17.2--22: Bassoon 1 written in by second hand in FS and part

25--28: Bassoon 1 written in by second hand in FS and part. Original Bassoon 2 part crossed out and replaced with ``col 1'' in part and FS.

37: Dynamics only in part. Adding tenuto marks like Flute 1

74: 8\textsuperscript{th} note A♯\textsubscript{2} added to Bassoon 1 in part

78: Bassoon 2 G♯\textsubscript{3} in FS. C♯\textsubscript{3} in part. Tenor clef likely missing in part

118--126: This section is absent in FS and marked ``tacet'' in part

143--145: This section for Bassoon 2 is only in the part. Using dynamics and articulation of Clarinets

153--156: \dynmark{\p} in FS. ``\dynmark{\mf} \textit{bien accentué}'' with accent on every note in part

161--165: Bassoon 1 ossia as in part

169--171: Bassoon 1 only in part (copy of Bass Clarinet Solo)

\underline{Sarrusophone}

130: The held note ends with a dotted half + 8\textsuperscript{th} note in FS, continues for a quarter note into 131 in part.

\underline{Horns 1 and 2}

Originally written with bass clef sounding up a fourth (treble down a fifth)

17: Horn 2 quarter note changed to 8\textsuperscript{th} note in part

38--45: Ossia as in part

70--74: This section for Horn 1 is crossed out in part and FS.

72: Horn 2 F♯ in FS overwriting an explicit ♮. Explicit F♮ in part. Concert B♭ and B♮ are both present in other instruments.

77--78: In the part, Horn 1 starting with at 77.1½ and Horn 2 starting with at 77.2½ are written a step higher than FS. In FS, 77.3 has text ``do mi♭ fa'' (C E♭ F) written above. These corrected pitches in the part match the Violins I.

83--87: This chord is held 1 measure less in part. 1 measure is crossed out and 1 measure rest is added afterwards.

89: ``sans sourdines'' only in part. Explicit ``sourdines'' in FS and part at 128

93: \dynmark{\sF} and decrescendo only in part

95--97: In part, 95 crescendo is changed to decrescendo, the whole note at 96 is only an 8\textsuperscript{th} note and 97 is crossed out.

128--129: In part, a slur is added over this whole section, ``\textit{expressif}'' is added, and all accents except on the half note are crossed out.

130 (146, 155): ``ôtez sourdines'' only in part

153--158: Ossia as in part

158.3: ♯ on C[♯] likely missing in FS. ♯ is added by a second hand to the C♯ in the Horn 3 part.

161--164: In the part, Horn 2 plays what Horn 3 has in FS. 164.3 E\textsubscript{5}s changed to C\textsubscript{5}s.

175--177: This note for Horn 2 is crossed out in part and FS.

\underline{Horns 3 and 4}

Originally written with bass clef sounding up a fourth (treble down a fifth)

15--18: Horn 3 plays unis with Bassoon 1 in part, crossed out. 19--20 written as cue

70--73: This section for Horn 3 is crossed out in FS and part.

72: See note for Horn 2

74--77: Idiosyncratic dynamics in part: crescendo on each phrase separately, \mbox{\dynmark{\p} $\rightarrow $ \dynmark{\f}}, \mbox{\dynmark{\mf} $\rightarrow $\dynmark{\ff}}, and then simile assumed. Using dynamics of rest of orchestra/Horns 1 and 2

75.2: Horn 4 Es instead of Fs in part, likely an error

77--78: See note for Horn 2

80: ``8a'' with extension line is written below the first 2 notes of Horn 4 in FS. ``8a'' without extension line is written below and before the first note in part. Playing this section an octave lower or higher is dubious.

83--88: This chord is held 2 measures less in part. 2 measures crossed out but only 1 measure rest is added afterwards, leaving 1 measure unaccounted for.

89: ``ôtez sourdines'' only in part

93: \dynmark{\sF} and decrescendo only in part

96: In Horn 3 part, this whole note is only a dotted quarter note and the next measure is crossed out.

96--100: Horn 4 part crossed out in part

97--100: Change(s) to bass clef at the end of 96 but the next system has treble clef (97 in FS and 98 in part). Bass clef reading matches Cellos and is used in this edition.

128: \dynmark{\mf} in FS. \dynmark{\ff} in part

130 (146, 153): ``ôtez sourdines'' only in part

153--158: Horn 3 ossia as in part

161--164: See note for Horn 2

\columnbreak

\underline{Trumpets 1 and 2}

50--51, 52--53: FS is very difficult to read. Trumpet 2 appears to have only a dotted quarter E♯\textsubscript{5} at 50 and 52, and dotted half note F♮\textsubscript{4} at 51 and 53. In this edition, this section is written as it is in the part, which matches other instruments better; it is a correction by a second hand covering the original.

66--70: Trumpet 2 an octave higher (starting with E\textsubscript{5}) in part. Dynamics only in part. 70 has B♭4 leading into Trumpet 1 Solo that is only in part.

70--73: This section for Trumpet 1 only in part.

82--85: Players unspecified in FS but has singular ``sourdine.'' Section is included in part for Trumpets 1 and 2, with the Trumpet 2 part written in second hand.

83--85: For Trumpet 1, this note lasts until an 8\textsuperscript{th} note at 84. For Trumpet 2, this note lasts only an 8\textsuperscript{th} note.

128: ``Sans sourdines'' is only in part, but next entrance has ``sourdines'' in both FS and part.

128--129: In part, slurs are added over each phrase, ``\dynmark{\mf} \textit{expressif}'' is added (\dynmark{\f} still present), and all accents except on the half notes are crossed out.

128--131: Trumpet 2 may need Trumpet in B♭ to play F\textsubscript{3}s and E\textsubscript{3}.

149: Trumpet 2 F♯ in FS that is corrected to G♯ in part. G♯ matches Violins I.

153: FS has divisi of B\textsubscript{3}, E\textsubscript{4}, B\textsubscript{4}. Trumpet 2 slur seems to lead to B\textsubscript{3}. Both Trumpets 2 and 3 have B\textsubscript{3} in part. Trumpet 3 part is most likely a misreading; it is written in by a second hand.

171--173: This section for Trumpet 1 only in part

175--178: This section crossed out in part

\underline{Trumpet 3}

153: See note for Trumpet 2. ``sourdine'' presumably missing to match other Trumpets.

161--167: FS has Trumpet 3 col 1, but part has independent part, written in by second hand.

\underline{Trombones 1 and 2}

66--69: Trombone 2 ``sans nuances'' only in part. Hairpins not crossed out but those at 68--69 are. It is unclear in FS whether the hairpins apply to Trombone 2; hairpins are doubled for Trombone 3 and Tuba but not Trombones 1 and 2.

83--85: This chord is held a measure shorter in part, until an 8\textsuperscript{th} note at 84.

128: ``Sans sourdines'' is only in part, but next entrance has ``sourdines'' in both FS and part.

128--129: In part, ``\textit{expressif}'' is added, and all accents except on the half note are crossed out.

149--153: FS has bass clef but part has tenor (correction in second hand). Tenor clef is more likely.

165--166: Slurs from 165.1--166.1, 166.3-- in part. Slur only from 166.1-- in FS. Using slurring consistent with other trombones

169--178: Ossia as in part (original of FS crossed out)

\underline{Trombone 3}

62--63, 64--65: Crescendos marked from \dynmark{\pp} to \dynmark{\p} in part. It seems dubious for Trombone 3 to be so much softer than the rest of the orchestra.

67, 69: B♭s in FS, unison with Horn 1 and Trombone 1. B♮s (explicit at 67) in part, octave with English Horn

83--85: This chord is held a measure shorter in part, until an 8\textsuperscript{th} note at 84.

128: ``Sans sourdine'' is only in part, but next entrance has ``sourdine'' in both FS and part.

128--129: In part, slur added over 128, ``\textit{expr}'' is added, and all accents except on the half notes are crossed out. Decrescendo at 129 in part

161--165: Part has what the Tuba has in FS (original written over in second hand).

171--173: Ossia as in part

\underline{Tuba}

62--63, 64--65: Crescendos marked from \dynmark{\p} to \dynmark{\mezzopiano} in part. It seems dubious for Tuba to be so much softer than the rest of the orchestra.

128: ``Sans sourdine'' is only in part, but next entrance has ``sourdine'' in both FS and part.

128--129: Accents crossed out and slurs only in part

161--165: Ossia as in part

171--173: Ossia as in part

\underline{Timpani}

65: Crescendo in FS. Crossed-out decrescendo in part

78--82: Crossed out in FS and part

79--80: ♭s presumably missing on D♭s in FS and part

87: Crossed out in part

\underline{Percussion}

Labeled as ``Cymbales'' but it is more likely suspended cymbal is intended.

26: Beat 1 Bass Drum is only in part

77--78: Cymbal adding crescendo to \dynmark{\ff}

78: Tam-tam \dynmark{\p} in FS. \dynmark{\f} with second-hand \dynmark{\fff} in part.

78--86: Bass Drum ossia as in part

90--98: Bass Drum ossia as in part

160: Cymbal only in part

161--167: Bass Drum ossia as in part

\underline{Celesta}

107: Last note has an F♯ instead of an E in FS

112--125: Ossia as in part

175--176: Ossia as in part

\columnbreak

\underline{Harp}

50: This measure only in part

81, 83--85: These measures crossed out in FS and part

104--107: D♯s tied only in FS, and it is unclear whether the Es are tied in FS. No ties in part

104--111: The lower staff Cs are only in the part.

111: A4 likely missing in first chord based on Celesta. (Rapid pedal changes do not seem to be a concern.)

122.2, 124.2: Part has D♮\textsubscript{4}s where there are F♮4s in FS. D♮\textsubscript{4}s would simplify the pedaling required.

\underline{Violins I}

Numerous articulation differences between FS and part. Generally articulation of part is used as it is more detailed.

18--22: ``div.'' is not written in FS or part, but system label at 21 in FS has \mbox{``div. en 2''} written. ``div.'' written at 23 in FS and part

48--49: Divisi with lower part tremeloing the same melody crossed out in FS and part

58: G\textsubscript{3} of triple stop crossed out in part

73.3: The lower divisi run starts on F\textsubscript{4} in part (i.e. the F\textsubscript{5} starting note of FS is excluded).

74--77: Div. with lower part tremeloing the same melody crossed out in FS and part

82: ``solo'' crossed out in part. The staff label in FS is unclear: ``V\textsuperscript{on} solo (line break) [illegible] autres 1\textsuperscript{o} V\textsuperscript{s}.''

96: The dynamics here are inconsistent between the strings. In FS Violins I and II have ``\dynmark{\sF} \textit{soutenu}'', Cellos are unmarked, and Double Basses have a crescendo. In the parts, Violins II have an \dynmark{\sF}\dynmark{\p} with crescendo (seemingly contradicting ``\textit{soutenu}''), Cellos have \dynmark{\p} with a crescendo, and Double Basses have a crescendo.

108--115: Solo Violin tacet in FS, with div. 1 in part (tacet indication crossed out).

118--125: Solo originally an octave higher in FS and part

125: Solo originally dotted half tied into 8\textsuperscript{th} note in next measure. Reduced to half note in both FS and part (with lowered octave)

\underline{Violins II}

Numerous articulation differences between FS and part. Generally articulation of part is used as it is more detailed.

15: Div. 1 and 2 flipped between FS and part

28--37: Tenuto marks and slurs only in part, besides a few tenuto marks that have been marked with double tenutos in this edition

42: sul G, likely intended for the melody, is impossible to play as the double stop involves the G string.

96: See note for Violins I

112--126: Originally arco with slurs over 2 measure groups. 119, 120, and 126 have an additional 8\textsuperscript{th} note tied from the previous measure that is crossed out. Changed to pizz in both FS and part, but arco is also added in at 119. See the Violin II ossia supplement for this original version.

126: The first note of all divisions in this measure seems entirely crossed out in FS. Only the 2\textsuperscript{nd} and 3\textsuperscript{rd} division are crossed out in part. The part seems more consistent with the changes in the previous measures.

146--152: Various discrepancies in dynamics between FS and part. Using combination of Violins II and Violas dynamics.

\underline{Violas}

Numerous articulation differences between FS and part. Generally articulation of part is used as it is more detailed.

10: Final note of upper divisi erased in part. This was likely only intended for the Solo player.

12.3: This section is written with the Solo and upper divisi on the upper staff (written in a second hand; what is visible of the original appears to be the same content omitting the Solo), and the lower divisi on the lower staff. C\textsubscript{4} in the divisi upper staff in FS seems to contradict the ``unis'' written here. The C\textsubscript{4} is not present in the part.

15: Unclear divisi in this measure. In FS, the upper staff only has a single A♭\textsubscript{3} quarter note, with 3 F\textsubscript{3}+A♭\textsubscript{3} quarter notes on the lower staff. In part, this is written on one staff with the divisi seemingly continuing from measure 13.

25--28: In part, the div. en 2 seems to have originally been div. en 3 with the double stops split.

28--37: Tenuto marks and slurs only in part, besides a few tenuto marks that have been marked with double tenutos in this edition

49: \dynmark{\sF}s only in part, on each beat. Unclear whether the \dynmark{\sF}s apply to both parts of the divisi

96--97: Copying dynamics from English Horn. This Solo is written by a second hand in FS and was originally just an English Horn cue in the part. See also note for English Horn.

112--116: In FS, this is written as in the ossia. However, the lower tremolos appear to be crossed out, and ``arco'' is missing if the tremolos were intended. In part, this is changed to how it is written in this edition. A note in a second hand in the FS seems to indicate this change.

117--121: (Continuing from previous note) In FS, upper divisi seems to have had the quarter note alternating tremolos crossed out and 8\textsuperscript{th} note beams added. Lower tremolos are no longer crossed out. Part continues same manner of playing as previously.

122--125: (Continuing from previous note) FS has what the part has. It seems likely that pizz. indication for the lower part is missing.

145: Dotted half changed to half + 8\textsuperscript{th} in part.

146--152: See notes for Violins II

147: F♭\textsubscript{5} in FS should likely be E♭\textsubscript{5} as in part.

152: Change to alto clef for Solo missing in FS

\columnbreak

\underline{Cellos}

Numerous articulation differences between FS and part. Generally articulation of part is used as it is more detailed.

19--20.1: Tie only in part. Lack of accent at 20.1 makes it seem plausible it is missing in FS.

20.3--21: Lower divisi tie only in FS

42, 44: Black noteheads on the double stops seem to be additions by a second hand. Unclear if black notehead at 44 crossed out in FS

67: Upper divisi C in FS, C♭ in part. Trio version does not match either.

69: Upper divisi C, D♭ in FS; C♭, D in part. Lower divisi has D♭, A♭, C in FS; D, A♭, C♭ in part. Trio version does not match either.

72--73: FS has bass clef (without previous courtesy clef) but tenor clef as in part is more likely (change to bass clef missing at 74).

86: ``solo'' crossed out for ``à 2'' (2 Cello Soli) in part and FS

96: See note for Violins I

97: E♭ in FS. E♮ (explicit ♮) in part is likely an error.

97--98: Upper and lower divisi switched between part and FS

113--123: This section of the the Solo in FS is overwriting a difficult to read original. Only the new version is in part.

118--125: The dynamics in the part for the Soli are much more precise than in FS.

126.3--128: Non-solo is ``tous'' in FS, but still Soli 5 cellos in part, with 1.3 playing the the written upper divisi and 2.4.5 playing the written lower divisi.

144--148: Divisi sections are switched in part, likely an error

148: Unis in part, with both divisi parts playing the last 8\textsuperscript{th} note

160.3--164: Starting from the 2\textsuperscript{nd} triplet 8\textsuperscript{th} (C), the part is an octave higher than FS.

165--169: Original part in FS has been crossed out, and is largely illegible. It is not present in the part and has been reconstructed from the trio version for the ossia.

\underline{Basses}

Numerous articulation differences between FS and part. Generally articulation of part is used as it is more detailed.

28--37: Tenuto marks and slurs only in part, besides a few tenuto marks that have been marked with double tenutos in this edition

46--47: Upper divisi tremolo added by second hand to FS and part. ``div.'' is not written in FS until 48.

48: Lower divisi tremoloing only in part

70--73: This section written by a second hand in FS, covering original (part matches new version).

78--82: ``unis'' in FS. The lower divisi has been added by a second hand.

83--88: Likely supposed to be only half (div.) to be consistent with the following measures, but there is no indication of such in FS or part.

96: See note for Violins I

98: In FS, this is played by the Basses besides the 1\textsuperscript{er} pupitre. In part, this is played by the 1\textsuperscript{er} pupitre.

139--142: Upper divisi crossed out in part. Replaced with unis

154--160: This section is added in by a second hand.

\end{multicols}

\end{hangparas}

\end{document}
